\documentclass[10pt, a4paper,english,spanish]{beamer}
\usepackage[utf8]{inputenc} % para poder usar tildes en archivos UTF-8
\usepackage[spanish]{babel} % para que comandos como \today den el resultado en castellano
\title[Presentación]{Presentación}
\subtitle{Seguridad inform\'atica} % Opcional
\author{Alis Yamil, De Rocco Federico y Muchinik Francisco}
\institute{UBA} % Opciona
\begin{document}
\begin{frame}
\titlepage
\end{frame}

\begin{frame}
\frametitle{Contenido}
\begin{itemize}
\item Introducción
\item Protección de logs
\item MAC
\item Forward Integrity
\item Nuestra implementación
\item Alternativas
\end{itemize}
\end{frame}

\begin{frame}
\frametitle{Introducción}
\begin{itemize}
\item Log: Es un registro que contiene los eventos ocurridos en un sistema.
\begin{itemize}
\item Esta información puede ser utilizada para detectar la existencia de ataques.
\item Un atacante podría evitarlo modificando estos registros.
\item Auditoría de logs: Método que permite averiguar si los logs fueron alterados.
\end{itemize}
\end{itemize}
\end{frame}

\begin{frame}
\frametitle{Protección de logs}
\begin{itemize}
\item Trusted Computing Base (TCB): Es el componente responsable de realizar el logging. Si se conserva la integridad del mismo,
los registros deberían ser seguros. Problema: no siempre están libres de bugs que permitan al atacante obtener
permisos.
\item Remote logging: Consiste en enviar los registros a otros equipos que los resguarden. Con esta medida
el atacante deberá conocer y vulnerar todos estos equipos para poder ocultar el ataque.
\item Imprimirlos: Una forma clásica de proteger los logs era imprimirlos de forma continua y ordenada.
Problemas:
\begin{itemize}
\item El sistema que se use para la impresión debe darle prioridad a los logs para que la impresora no se vea sobrecargada.
\item Un análisis de actividades sospechosas es mucho más difícil.
\item Se puede comprometer la impresora o las impresiones.
\end{itemize}
\item Write Once Read Multiple (WORM): Son discos ópticos de poco ancho de banda de escritura. Dichos discos
se extraen una vez que están llenos. Problema: se puede comprometer la integridad si se sustituyen uno o más discos.
\end{itemize}
\end{frame}

\begin{frame}
\frametitle{MAC}
\begin{itemize}
\item Se calcula un MAC para cada log usando un secreto. Esto permite impedir que, en desconocimiento de ese secreto,
el atacante pueda modificar el log. Si lo hace tendría que recalcular el MAC. Problemas:
\begin{itemize}
  \item En ausencia de un continuo uso de remote logging este mecanismo no asegura la integridad de los
  registros viejos. Esto se debe a que el secreto es usado en el sistema que realiza el logging. Si este
  se compromete, también se obtiene el secreto.
  \item El remote logging tampoco es perfecto ya que la seguridad de estos hosts pasa a ser crítica.
\end{itemize}
\end{itemize}
\end{frame}


\begin{frame}
\frametitle{Forward Integrity}
\begin{itemize}
\item Básicamente queremos generar un MAC para cada log que no pueda ser modificado sin quedar en evidencia
aunque el sistema que genera los logs sea comprometido.
\item La idea es que el sistema no repita la clave utilizada en los pasos anteriores. Además, una vez calculado
el MAC se debe descartar dicha clave para evitar que se la pueda obtener.
\item Una forma de asegurar esta propiedad es generar la clave actual en base a la inmediatamente anterior.
Tenemos para el log número i la clave $K_{i}$ que se obtiene aplicándole una función no reversible a
$K_{i-1}$. Una vez calculada $K_{i}$, se borra $K_{i-1}$.
\item Si el atacante obtiene control del sistema en el momento que el siguiente log es el número i
obtendría la clave $K_{i}$. Con la cual no puede obtener la $K_{i-1}$ ni ninguna de las anteriores que
lo delatan.
\end{itemize}
\end{frame}


\begin{frame}
\frametitle{Forward Integrity}
\begin{itemize}
\item Este sistema deja implícito que debe existir un secreto inicial y si este se compromete se pierde la
seguridad. Por esto el secreto inicial debe estar guardado en un lugar seguro.
\item A la hora de validar el estado de los registros se debe aportar el secreto inicial. Se le aplica a
este la función no reversibles y se calcula el MAC para el primer log. Se repite el proceso para los
siguientes.
\item El componente que realice la verificación debe estar separado el que genera los logs.
\item Es igualmente válido utilizar una clave aleatoria para cada entrada en el registro. Problema:
Se deben guardar todas las claves utilizadas(una por log).

\end{itemize}
\end{frame}

\begin{frame}
\frametitle{Nuestra implementación}
\begin{itemize}
\item Utilizamos un secreto inicial, elegido por el administrador del equipo, para iniciar el servidor syslog
y realizar la verificación de la integridad de los logs en cualquier momento.
\item Los registros de eventos se guardan en un ".log" de solo lectura para un usuario normal.
\item Cada verificación realiza el procedimiento descripto para todos los logs en el registro.
\end{itemize}
\end{frame}

\begin{frame}
\frametitle{Nuestra implementación}
%Imagenes
\end{frame}

\begin{frame}
\frametitle{Nuestra implementación}
Demo
\end{frame}

\begin{frame}
\frametitle{Alternativas}
\begin{itemize}
\item Sistema de clave pública
%Imagen de clave publica y verificación
\end{itemize}
\end{frame}

\begin{frame}
\frametitle{Alternativas en la verificación}
\begin{itemize}
\item Existen dos alternativas a la hora de validar los logs:
\begin{itemize}
\item Validarlos todos(nuestra opción).
\item Validar solamente los que no fueron validados anteriormente.
\end{itemize}
\end{itemize}
\end{frame}

\end{document}
